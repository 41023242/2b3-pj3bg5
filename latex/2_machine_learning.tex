\chapter{group}

\section{列出所有成員資料}
<h1>group</h1>
<p></p>
<p><button id="add1to100">亂數</button></p>
<!-- 導入 brython 程式庫 -->
<p>
<script src="/static/brython.js"></script>
<script src="/static/brython_stdlib.js"></script>
</p>
<!-- 啟動 Brython -->
<p>
<script>// <![CDATA[
window.onload=function(){
brython({debug:1, pythonpath:['/static/','./../downloads/py/']});
}
// ]]></script>
</p>
<p><!-- 導入 FileSaver 與 filereader --></p>
<p>
<script type="text/javascript" src="/static/ace/FileSaver.min.js"></script>
<script type="text/javascript" src="/static/ace/filereader.js"></script>
</p>
<p><!-- 導入 ace --></p>
<p>
<script type="text/javascript" src="/static/ace/ace.js"></script>
<script type="text/javascript" src="/static/ace/ext-language_tools.js"></script>
<script type="text/javascript" src="/static/ace/mode-python3.js"></script>
<script type="text/javascript" src="/static/ace/snippets/python.js"></script>
</p>
<p><!-- 導入 gearUtils-0.9.js Cango 齒輪繪圖程式庫 -->
<script src="/static/Cango-24v03-min.js"></script>
<script src="/static/gearUtils-09.js"></script>
<script src="/static/SVGpathUtils-6v03-min.js"></script>
<script src="/static/sylvester.js"></script>
<script src="/static/PrairieDraw.js"></script>
</p>
<p><!-- 請注意, 這裡使用 Javascript 將 localStorage["kw_py_src1"] 中存在近端瀏覽器的程式碼, 由使用者決定存檔名稱--></p>
<p>
<script type="text/javascript">// <![CDATA[
function doSave(storage_id, filename){
    var blob = new Blob([localStorage[storage_id]], {type: "text/plain;charset=utf-8"});
    filename = document.getElementById(filename).value
    saveAs(blob, filename+".py");
}
// ]]></script>
</p>
<p>
<script type="text/python3">// <![CDATA[
from browser import document as doc
import ace
# 清除畫布
def clear_bd1(ev):
    bd = doc["brython_div1"]
    bd.clear()
Ace1 = ace.Editor(editor_id="kw_editor1", console_id="kw_console1", container_id="kw__container1", storage_id="kw_py_src1" )
# 從 gist 取出程式碼後, 放入 editor 作為 default 程式
def run1():
    # 利用 get 取下 src 變數值
    try:
        url = doc.query["src"]
    except:
        url = "https://gist.githubusercontent.com/41023242/ae8d10daf540e1640cfbc32f1114e871/raw/501e3ad784fd19222e98c3b3705e748bfdab2cec/mmmm"
    prog = open(url).read()

    # 將程式載入編輯區
    Ace1.editor.setValue(prog)
    Ace1.editor.scrollToRow(0)
    Ace1.editor.gotoLine(0)
    # 直接執行程式
    #ns = {'__name__':'__main__'}
    #exec(prog, ns)
    # 按下 run 按鈕
    Ace1.run()

# 執行程式, 顯示輸出結果與清除輸出結果及對應按鈕綁定
doc['kw_run1'].bind('click', Ace1.run)
doc['kw_show_console1'].bind('click', Ace1.show_console)
doc['kw_clear_console1'].bind('click', Ace1.clear_console)
doc['clear_bd1'].bind('click', clear_bd1)
# 呼叫函式執行
run1()
// ]]></script>
</p>
<p><!-- 亂數 開始 -->
<script type="text/python3">// <![CDATA[
from browser import document as doc
import ace

# 清除畫布
def clear_bd1(ev):
    bd = doc["brython_div1"]
    bd.clear()

# 利用 ace 中的 Editor 建立 Ace2 物件, 其中的輸入變數分別對應到頁面中的編輯區物件
Ace2 = ace.Editor(editor_id="kw_editor1", console_id="kw_console1", container_id="kw__container1", storage_id="kw_py_src1" )

add1to100_url = "https://gist.githubusercontent.com/41023242/e9a398ddf001f2422be93e06ca89638d/raw/6cfafceb8155ca8f8c1e87f568aa92e3d3bb1763/ggg"

# 從 gist 取得程式碼
add_src = open(add1to100_url).read()
def add(ev):
    Ace2.editor.setValue(add_src)
    Ace2.editor.scrollToRow(0)
    Ace2.editor.gotoLine(0)
    Ace2.run()

# id 為 "亂數" 的按鈕點按時, 執行 add 方法
doc["add1to100"].bind('click', add)
// ]]></script>
</p>
<p><!-- 亂數 結束--></p>
<!-- editor1 開始 -->
<p><!-- 用來顯示程式碼的 editor 區域 --></p>
<div id="kw_editor1" style="width: 600px; height: 300px;"></div>
<p><!-- 以下的表單與按鈕與前面的 Javascript doSave 函式以及 FileSaver.min.js 互相配合 --></p>
<p><!-- 存擋表單開始 --></p>
<form><label>Filename: <input id="kw_filename" placeholder="input file name" type="text">.py</label> <input onclick="doSave('kw_py_src1', 'kw_filename1');" type="submit" value="Save"></form>
<p><!-- 存擋表單結束 --></p>
<p></p>
<p><!-- 執行與清除按鈕開始 --></p>
<p><button id="kw_run1">Run</button> <button id="kw_show_console1">Output</button> <button id="kw_clear_console1">清除輸出區</button><button id="clear_bd1">清除繪圖區</button><button onclick="window.location.reload()">Reload</button></p>
<p><!-- 執行與清除按鈕結束 --></p>
<p></p>
<p><!-- 程式執行 ouput 區 --></p>
<div style="width: 100%; height: 100%;"><textarea autocomplete="off" id="kw_console1"></textarea></div>
<p><!-- Brython 程式執行的結果, 都以 brython_div1 作為切入位置 --></p>
<div id="brython_div1"></div>
<!-- editor1 結束 --><hr><!-- ########################################## -->
<p></p>
\newpage
